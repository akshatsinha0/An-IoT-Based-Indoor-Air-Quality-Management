% !TEX program = lualatex
\documentclass[12pt]{report}

% Font and layout
\usepackage{fontspec}
\IfFontExistsTF{Times New Roman}{\setmainfont{Times New Roman}}{\setmainfont{TeX Gyre Termes}}
\usepackage[a4paper,margin=1in]{geometry}
\usepackage{setspace}
\onehalfspacing
\usepackage{titlesec}
\titleformat{\section}{\bfseries\fontsize{14pt}{16pt}\selectfont}{\thesection.}{0.5em}{}
\titleformat{\subsection}{\bfseries\fontsize{13pt}{15pt}\selectfont}{\thesubsection}{0.5em}{}
\usepackage{hyperref}
\hypersetup{colorlinks=true,linkcolor=black,urlcolor=blue}
\usepackage{graphicx}
\usepackage{array}
\usepackage{enumitem}
\setlist{noitemsep,topsep=2pt}

% Convenience macros
\newcommand{\course}{BCSE401L -- Internet of Things}
\newcommand{\projecttitle}{An IoT-Based Indoor Air Quality Management System}
\newcommand{\institute}{Vellore Institute of Technology Vellore -- 632014, INDIA}
\newcommand{\dept}{Department of Computer Science \& Engineering}
\newcommand{\authors}{ARNAV SINHA -- 22BCE0830\\AKSHAT SINHA -- 22BCE2218}
\newcommand{\guides}{USHUS\\ELIZEBETH\\ZACHARIAH}
\newcommand{\reportdate}{8, August 2025}

\begin{document}

% Cover Page
\begin{titlepage}
\centering
\vspace*{1cm}
{\large A\\[6pt] Report Submitted\\[6pt] for}\\[12pt]
{\bfseries\Large Project Review 1}\\[10pt]
{\itshape on the project entitled}\\[4pt]
{\bfseries\LARGE \projecttitle}\\[8pt]
{\large Submitted in partial fulfillment of the requirements for the course}\\[4pt]
{\Large \course}\\[12pt]
{by}\\[6pt]
{\large \authors}\\[12pt]
{Under the guidance of}\\[6pt]
{\large \guides}\\[12pt]
\vfill
\IfFileExists{vit_logo.png}{\includegraphics[width=3cm]{vit_logo.png}\\[6pt]}{}
{\dept}\\[4pt]
{\institute}\\[6pt]
{\reportdate}
\end{titlepage}

\pagenumbering{roman}
\tableofcontents
\cleardoublepage
\pagenumbering{arabic}

% 0. Abstract
\section*{Abstract}
\addcontentsline{toc}{section}{0. Abstract}
% (empty for now if not provided)

% 1. Introduction
\section{Introduction}
Our project is about creating a software-based indoor air quality (IAQ) monitoring system using Internet of Things (IoT) technologies. The system tracks indoor air parameters like temperature, humidity, and gas levels in real time and shows them through a web interface. This project matters because people spend most of their time indoors and poor air can affect health, mood, and productivity.

% 2. Motivation
\section{Motivation}
Clean indoor air is essential for good health at home, work, hospitals, and elderly care. Since air pollutants are invisible, we want to make IAQ visible and actionable. Affordable sensors and smart software can help people without complex hardware.

% 3. Scope of the Project
\section{Scope of the Project}
\begin{itemize}
  \item Build a software-only IAQ system that collects, processes, and visualizes indoor air quality data.
  \item Let users access data through a web portal.
  \item Include real-time alerts when air quality drops below safe levels.
  \item Store and display historical trends for better air quality management.
\end{itemize}
It aims to help building managers, healthcare workers, and residents.

% 4. Literature Review
\section{Literature Review}
People spend most of their time indoors, so indoor air strongly affects health and well-being. Earlier IAQ systems used low-cost sensors with gateways and web/mobile apps for viewing data. Later work added machine learning to classify and predict IAQ, focused systems for particulate matter (PM), and even active neutralization for dust and temperature. Reviews also show common choices like Arduino/ESP8266, Wi‑Fi links, cloud storage, and alerts.

\subsection{Foundational work}
An early Ambient Assisted Living IAQ system used Arduino sensor nodes, an ESP8266 gateway, and a web/mobile front end. It measured temperature, humidity, CO, CO\textsubscript{2}, and light. Strengths were low cost and modularity; limits included sensor accuracy and power use.

\subsection{Predictive analytics}
A COVID‑19 era system monitored multiple gases and PM2.5 and used neural networks and LSTM to classify and predict IAQ with high accuracy, enabling proactive action.

\subsection{Specialized PM monitoring}
The iDust system focused on PM10/PM2.5/PM1.0 with a PMS5003 sensor and a web dashboard, showing medical value of historical PM exposure and easy Wi‑Fi setup.

\subsection{Active neutralization}
Another system added control: a sprayer for dust and an exhaust fan for temperature, based on national thresholds, moving from monitoring to action.

\subsection{Common trends and challenges}
Trends include open‑source hardware/software, mobile and web dashboards, and alerts. Persistent issues are sensor calibration, power use of Wi‑Fi, scalability, long‑term drift, privacy/security, and limited integration with building systems.

% 5. Research Gap
\section{Research Gap}
\begin{itemize}
  \item Low and inconsistent field calibration for low‑cost sensors reduces trust in data.
  \item Wi‑Fi power use hurts long‑term battery deployments; low‑power options have range limits.
  \item Hard to deploy many nodes for complete coverage; maintenance grows with scale.
  \item Sensor lifetime and drift hurt long‑term accuracy.
  \item Privacy, confidentiality, and security of collected data need stronger solutions.
  \item Limited integration with HVAC/building systems for automated control.
\end{itemize}

% 6. Novelty / Innovation
\section{Novelty / Innovation}
We add a Central Pollution Control Board (CPCB) National Air Quality Index aligned PM2.5 alerting with time‑in‑zone exposure:
\begin{itemize}
  \item Map each PM2.5 reading to CPCB categories: Good, Satisfactory, Moderately Polluted, Poor, Very Poor, Severe; color‑code the dashboard to match local standards.
  \item Keep running counters of minutes/hours spent in each category for the last day and week to capture prolonged exposure.
  \item Software‑only pipeline: compute sub‑index per reading, update exposure counters, and store both with raw data for queries and the Streamlit dashboard.
  \item Works with OpenAQ feeds or simulated dataset replay and follows CPCB breakpoints, so it is locally relevant in India.
\end{itemize}

% 7. Problem Statement
\section{Problem Statement}
Indoor places can have hidden pollutants that harm health. Many monitoring systems are expensive or hardware‑heavy. We build a low‑cost, software‑based IAQ system that uses simulated or API data, shows it in a simple UI, and sends alerts when limits are crossed. It does not use real sensors; accuracy depends on input data quality.

% 8. System Design & Architecture
\section{System Design \& Architecture}
The software has four parts: a data generator or API fetcher, a FastAPI backend with storage, an analytics layer that computes CPCB PM2.5 sub‑index and exposure time, and a Streamlit dashboard for live views and alerts.\\
\IfFileExists{architecture.png}{\begin{center}\includegraphics[width=0.8\linewidth]{architecture.png}\end{center}}{}

% 9. Description of the Environment
\section{Description of the Environment}
Python with FastAPI and Streamlit. Key libraries: pandas, numpy, plotly. Storage uses SQLite/CSV for time‑series data. The app runs locally with Uvicorn for the API and Streamlit for the UI.

% 10. Metrics Used
\section{Metrics Used}
\begin{itemize}
  \item System responsiveness to data changes.
  \item Alert correctness vs. CPCB thresholds.
  \item Clarity and usability of the dashboard.
  \item Trend accuracy for simulated data.
\end{itemize}

% 11. Performance Evaluation
\section{Performance Evaluation}
% (empty for now)

% 12. Conclusion
\section{Conclusion}
IoT‑based IAQ systems evolved from basic monitoring to prediction and action. Challenges remain with calibration, power, scale, and security. Our software‑only system focuses on reliable analytics and CPCB‑aligned PM2.5 alerts with exposure tracking to support healthier indoor spaces.

% 13. Future Work
\section{Future Work}
Connect IAQ control with wider smart‑building functions, explore advanced AI (e.g., reinforcement learning), study long‑term reliability, link IAQ improvements to health outcomes, and strengthen privacy/security.

% 14. Reference
\section{Reference}
\begin{thebibliography}{9}
\bibitem{marques2016} Marques, G., \& Pitarma, R. (2016). An indoor monitoring system for ambient assisted living based on Internet of Things architecture. \emph{IJERPH}, 13(11), 1152.
\bibitem{saini2020} Saini, J., Dutta, M., \& Marques, G. (2020). Indoor Air Quality Monitoring Systems Based on Internet of Things: A Systematic Review. \emph{IJERPH}, 17(14), 4942.
\bibitem{mumtaz2021} Mumtaz, R., et al. (2021). IoT Based Indoor Air Quality Sensing and Predictive Analytic—A COVID‑19 Perspective. \emph{Electronics}, 10(2), 184.
\bibitem{pangestu2020} Pangestu, A., et al. (2020). The monitoring system of indoor air quality based on Internet of Things. \emph{ISITIA}, 95--100. IEEE.
\bibitem{marques2018} Marques, G., Ferreira, C. R., \& Pitarma, R. (2018). A system based on the Internet of Things for real‑time particle monitoring in buildings. \emph{IJERPH}, 15(4), 821.
\end{thebibliography}

% 15. Code
\section{Code}
Project repository: \url{https://github.com/akshatsinha0/An-IoT-Based-Indoor-Air-Quality-Management}

% 16. PPT for presentation (not part of the Report)
\section*{PPT for presentation (not part of the Report)}
% (placeholder)

\end{document}
